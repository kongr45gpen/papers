\documentclass[]{iac}
% To make the list of abbreviations and symbols. 
\usepackage[acronym]{glossaries}
\usepackage{hyperref}
\usepackage[citestyle=authoryear,bibstyle=authortitle,maxbibnames=1,maxcitenames=1]{biblatex}
\usepackage[dvipsnames]{xcolor}
\usepackage{enumitem}
\makeglossaries

\acsetup{use-id-as-short}

\DeclareAcronym{MBSE}{
    long = Model-Based Systems Engineering
}

\DeclareAcronym{CI}{
    long = Continuous Integration
}

\DeclareAcronym{CD}{
    long = Continuous Delivery
}

\DeclareAcronym{TDD}{
    long = Test-Driven Development
}

\DeclareAcronym{PCB}{
    long = Printed Circuit Board
}

\DeclareAcronym{COTS}{
    long = Commercial Off-The-Shelf
}

\DeclareAcronym{RF}{
    long = Radio Frequency
}

\DeclareAcronym{UL} {
    long = University of Luxembourg
}

\DeclareAcronym{SoC} {
    long = System on Chip,
}

\DeclareAcronym{IC} {
    long = Integrated Circuit
}



\newcommand\myshade{85}
\colorlet{mylinkcolor}{violet}
\colorlet{mycitecolor}{Turquoise}
\colorlet{myurlcolor}{Blue}

\hypersetup{
  linkcolor  = mylinkcolor!\myshade!black,
  citecolor  = mycitecolor!\myshade!black,
  urlcolor   = myurlcolor!\myshade!black,
  colorlinks = true,
}

\DeclareMathOperator{\E}{E}
\DeclareMathOperator{\prob}{p}
\DeclareMathOperator{\tr}{tr}

\newcommand{\etalia}{\textit{et al.}}
\newcommand*{\vectornorm}[1]{\left\|#1\right\|}
\newcommand*\rfrac[2]{{{}^{#1}\!/_{#2}}} % running fraction with slash - requires math mode.
\newcommand*\T{\mathsf{T}}

\addbibresource{references.bib}

\begin{document}

\IACpaperyear{22}
\IACpapernumber{D1,IPB,15,x71812}
\IACconference{73}
\IAClocation{Paris, France, 18-22 September 2022}
\IACcopyrightA{2022}{International Astronautical Federation (IAF)}

\title{Agile-Systems Engineering for sub-CubeSat scale spacecraft}

\IACauthor{Konstantinos~Kanavouras}{Affiliation, Country, email address}
\IACauthor{Andreas~Hein}{Affiliation, Country, email address}
\IACauthor{Maanasa~Sachidanand}{Affiliation, Country, email address}

\abstract{Space systems miniaturization has been increasingly popular for the past decades, with over 1600
CubeSats and 300 sub-CubeSat sized spacecraft estimated to have been launched since 1998. This trend
towards decreasing size enables the execution of unprecedented missions in terms of quantity, cost and
development time, allowing for massively distributed satellite networks, and rapid prototyping of space
equipment. Pocket-sized spacecraft can be designed from scratch in less than a year and can reach weights
of less than 10g, taking away the considerable costs and requirements typically associated with orbital
flight. However, while Systems Engineering methodologies have been proposed for missions down to
CubeSat size, there is still a gap regarding design approaches for picosatellites and smaller spacecraft,
which can exploit their potential for iterative and accelerated development. In this paper, we propose a
Systems Engineering methodology that abstains from the classic waterfall-like approach in favor of agile
practices, focusing on available capabilities, delivery of features and design ”sprints”. This methodology
originates from the software engineering discipline and shifts away from the typical system-subsystem-
component model, suggesting instead to specify desired capabilities which are directly linked to one or
more components. To account for the decrease in reliability due to the more ”relaxed” nature of this
approach, we also explore the degree of reliability and value added by simultaneously launching identical
designs, different designs, or a combination of both. This methodology allows quick adaptation to imposed
constraints, changes to requirements and unexpected events (e.g. chip shortages or delays), by making
the design flexible to well-defined modifications. 2 femtosatellite missions, currently under development
and due to be launched in 2023, are used as case studies for our approach, showing how miniature
spacecraft can be designed, developed and qualified from scratch in 6 months or less. Both missions
involve the attachment of a chip-sized satellite (“ChipSat”) into a larger spacecraft, either relying on
their host for communications and power or being completely independent. The proposed methodology
has applications in Earth orbits and beyond, bringing well-established design practices into the domain of
aerospace engineering and providing a well-structured approach to the creation of small-sized spacecraft,
which benefits from quick integration of past lessons learned with new technologies.}
\IACkeywords{maximum 6 keywords}{}{}{}{}{}

\maketitle
% Add list of symbols 
\printglossary[type=\acronymtype, title=Abbreviations]
\printglossary[title=Nomenclature]

\newpage
.

\pagebreak[4]
\newpage

\setlist[itemize]{leftmargin=*}

\section{Introduction}
    \begin{enumerate}
        \item ChipSat spacecraft
        \begin{itemize}
            \item An introduction to CubeSats and smaller spacecraft \autocite{hein_attosats_2019}
            \item Survey/review of any new missions/updates since 2019? \autocite{adams_theory_2020}
        \end{itemize}
        \item Systems Engineering for Space
        \begin{itemize}
            \item Reference to traditional approaches \autocite{shea_nasa_2017, ecss_ecss-e-st-10c_nodate}
            \item Reference to deviations to traditional approaches in experimental projects/startups/education (???)
        \end{itemize}
        \item Agile Systems Engineering approaches
        \begin{itemize}
            \item What is agile systems engineering, how it has been used \autocite{douglass_agile_2015, haberfellner_agile_2005}
            \item "Agile" Systems Engineering applied to CubeSats (citations)
        \end{itemize}
        \item Contribution of this paper
        \begin{itemize}
            \item An agile systems engineering framework for ChipSats
            \item Taking advantage of specific ChipSat capabilities \autocite{hein_attosats_2019}
        \end{itemize}
    \end{enumerate}
\section{Method}
\begin{enumerate}
\def\labelenumi{\arabic{enumi}.}
\item
  Concepts directly applicable from \cite{douglass_agile_2015}

  \begin{itemize}
  \item
    Ideas from Agile Manifesto \& principles
  \item
    Work $\to$ separable into parts and iterations
  \item
    Verify quality of work at end of each iteration (Test-Driven
    Development). Integrate continuously (instead of developing separate
    parts)
  \item
    Project Management: Dynamic planning, project risk reduction
  \end{itemize}
\item
  Use MBSE but with care

  \begin{itemize}
  \item
    Example: Behavior-driven development
  \item Automated generation of software based on models \autocite{perrotin_taste_2012,bychkov_using_2018}
%  \item
%    Try to find examples of misuse of MBSE (???)
  \end{itemize}
\item
  Technology reuse, but with care
\item
  Launching multiple spacecraft \autocite{hein_attosats_2019, adams_r-selected_2019}

  \begin{itemize}
  \item
    Selecting the ideal number N of spacecraft for better reliability. Maximisation of cost/success probability ratio

  \end{itemize}
\end{enumerate}
\section{Case Studies}
	\subsection{AI4Space}
 \begin{itemize}
\item Project/Interface/Payload description (Maanasa)
\item Systems Engineering
\begin{itemize}
    \item  Functioning/Shippable system from day N
    \item Multiple iterations/rewrites
    \item Test-Driven Development
    \item Use of MBSE (e.g. Python Construct)
    \item Continuous Integration
    \item Technology reuse
    \item Lessons-learned resuse
    \item Project Management
\end{itemize}
 \end{itemize}
			
	\subsection{ChipSat}
 \begin{itemize}
		\item Not sure how much we can include here, since we're still in early stages. Maybe make plans on the next steps?
		\item Functionality split into 3 spacecraft
		%\item Probability of mission success calculations
  \end{itemize}
\section{Discussion}
\begin{enumerate}
    \item Lessons Learned, Advantages observed
    \item Future Steps
\end{enumerate}
\section{Conclusion}

\vspace{3em}

\section*{Acknowledgements}
%- Check if need to acknowledge uni.lu HPC

\newpage
\printbibliography

\end{document}